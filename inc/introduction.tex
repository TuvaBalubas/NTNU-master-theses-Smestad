\chapter{Introduction}
\label{chap:introduction}
\q{What is the purpose of this document?}
The purpose of the current document is to provide an example of using the thesis template and a description on how to use \LaTeX. There are instructions that relate specifically to the template, and some which are generally useful for \LaTeX. 

\q{Where is the structure defined } 
The detailed typographical rules have been implemented in the
\texttt{ntnuthesis} \LaTeX\ document class, in the file 

This was updated by Simon McCallum.
The package has changed names and version number it is now called
\texttt{ntnuthesis}
v\ntnuthesisversion\ as of \ntnuthesisdate.

\q{what is a thesis about}
For many of you, this Master's thesis will be the most advanced academic document you ever write.  It needs to demonstrate both academic ability and clear thinking. You Master's should show that you are ready to lead other people, reflect more deeply, and have a professional attitude to your work and environment. 

\q{who cares}
When writing the thesis it is important to know who you are writing for. The target audience for this document is in layers:
\begin{enumerate}
    \item The marking committee
    \item Your supervisor
    \item Other students at the same level 
    \item Professionals \& Academics
    \item The general public.
\end{enumerate}

\section{Problem Description}
The chatbots of 2017 have not truly improved since the first chatbot ELIZA back in 1966. While the programming behind conversational interfaces are fast improving, making use of Artificial Intelligence (AI) and Natural Language Processing (NLP), a chatbot is still a form of weak-AI in which makes use of stimulus-response approach. Human interaction and conversation consists of a lot more than predicting user intentions, analyse keywords to define meaning, and then respond from a predefined set of responses. Therefore, no matter how accurate the chatbot is in predicting what humans actually mean, and how well it performs its tasks, they are not perceived as intelligent conversational entities. Predictions and recent research find chatbots to be a big part of an AI powered future, but recent reviews of chatbots have found them to be unintelligent and non-conversational (Stokke, 2017, Orf, 2017, Piltch, 2017, Vincent, 2017, Boutin, 2017). Piltch (2017) states that we should not be carried away by the positive outlook researchers presents in regards to the possibilities of advances in AI technology for chatbot technology, as the reality is that most chatbots are falling flat. Despite cautions and recent negative reviews, Forrester (2017) found that 56 \% of companies have implemented or are planning to implement a chatbot as part of the services they provide in the near future. JuniperResearch (2017) released a report in which they have found that chatbots will save companies \textdollar 8 billion in costs by 2022. Therefore, as the trends predict many benefits for companies implementing chatbots, are we forcing users to adopt technology which they find frustrating and useless? If the reviews find chatbot interactions as unintelligent, pointless, and not more effective than conducting a Google search or contacting a human customer service agent, what effects might this have on the user experience?

\section{Justification, Motivation and Benefits}
Ever since ELIZA, the goal of chatbot systems have been to pass the Turing Test, and convince humans that they are conversing with a human, not a machine \citep{McTear2016b}. However, available chatbots does not appear to possess human conversational skills, rather they perform as a machine by responding to user commands. This shows that although a chatbot is more flexible in regards to it being able to understand vast variations of the same command, it still functions as a command-based system. To be perceived as conversational agents, chatbots needs to be able to do a lot more than input-output. Therefore, understanding why chatbots are failing to be conversational and what we can do to allow users to perceive them as more than a computer, can have a great impact on improving the user experience. To understand how we can design chatbots that are perceived as more conversational, means that we must take into account many other factors of human interaction beyond only language understanding. Humans have bodies, feelings, emotions, different personalities and behaviours that all influence how we communicate, behave, and interact. Humans are extremely skilled social actors, and in order to design chatbots that too are skilled social actors we must understand the social cues that make up human interaction, and the personalities that drives them. 

This master thesis project will explore how we can build chatbots that offer a better user experience and are perceived as more conversational through focusing on the design of chatbot personalities. The thesis will understand how thinking about the chatbot as a character with a mind of its own, can benefit the design process and result in more conversational and improved user experience. In addition to this, chatbots are extensions of services provided by companies, therefore it is also important to design personalities that reflects the brand it represents. Through this topic the researcher hopes to add to the understanding of how designers can create better chatbot interfaces, and offer a framework to inform how designing chatbot personalities can benefit the design process as a whole. As trends show that companies are rapidly implementing chatbots as part of their services, consumers should not be forced to adopt solutions that does not improve the effectiveness, efficiency or satisfaction. This investigation would also benefit companies, as this knowledge can help ensure improved services for their customers, and consistent communication in its tone of voice through the chatbot.

\section{Research Questions}
The research questions and sub-questions to be addressed in the master thesis project are:

\begin{enumerate}
    \item How can we design chatbot personalities to guide the design process of a chatbot interface and create a better user experience?
    \subitem a) Which elements must be considered to inform a chatbot personality?
    \subitem b) What components needs to be in place for a chatbot personality to meet user needs and expectations?
    \item How can personality be used as a design variable to allow users to perceive chatbots as more than a computer?
        \subitem a) How can designers use personality to improve the conversational design of chatbots?
        \subitem b) How will the personality affect the user experience?
        \end{enumerate}

\section{Planned Contributions}
The results from exploring techniques to design user-centred chatbot personalities will be presented to define a UX framework that designers can use to build user centred chatbot personalities. This master thesis will produce a framework that entails different methods and techniques that designers can use to design a chatbot persona that meets user needs and is consistent with the brand's tone of voice. This framework will form a foundation to guide the design process of chatbots, towards improved user experience. This framework will inform what elements that needs to be considered to inform the type of personality that is appropriate, and which methods that could help design the personality to be in line with users' needs as well as the brand it represents. This framework will also inform how the chatbot personality can be used to guide and inform other aspects of the design of chatbots such as the conversation design. The framework will include methods to not only design, but evaluate design decisions in relation to the user experience, show how companies can extend and add value to their brands by providing a chatbot experience that is consistent with the brand's tone of voice.