\chapter{Introduction}
\label{chap:introduction}
The chatbots of 2018 have not truly improved since the first chatbot ELIZA created back in 1966. While recent advances in machine learning has contributed to fast improvements in Artificial Intelligence (AI) and Natural Language Processing (NLP) of conversational interfaces, chatbots are still not perceived as intelligent conversational actors. There has been an explosion of available chatbot agents online the past year, and they are now representing companies by directly providing services to their customers. Ever since ELIZA, the goal of chatbot systems have been to pass the Turing Test, and convince humans that they are conversing with a human, not a machine \citep{McTear2016b}. However, available chatbots does not appear to possess human conversational skills, rather they perform as machines that responds to user commands. The recent advances has contributed to chatbot systems becoming more flexible in regards to it being able to understand vast variations of the same command, and the implementation of cloud-based systems has allowed for an explosive growth of devices connected through the internet, also known as the Internet of Things (IoT). Therefore access to AI has become widespread, and through application programming interfaces (APIs), chatbots have access to vast amounts of information and knowledge through thousands of databases online. All this sounds promising, and explains in large part why chatbots have seen a rebirth recently, but all this does not matter if chatbots cannot live to the expectations of users.  Predictions find chatbots to be a big part of an AI powered future, but recent reviews have found them to be unintelligent and non-conversational \citep{stokke2017,orf2017,piltch2017,vincent2017,boutin2017}. \cite{piltch2017} states that we should not be carried away by the positive outlook researchers presents in regards to the possibilities of advances in AI for chatbot technology, as the reality is that most chatbots are falling flat. The fast implementation of chatbots have resulted in flawed interfaces that fails to predict the simplest of questions. Despite cautions and recent negative reviews, \cite{forrester2017} found that 57 \% of companies have implemented or are planning to implement a chatbot as part of the services they provide in the near future. \cite{juniper2017} released a report in which they have found that chatbots will save companies \textdollar 8 billion in costs by 2022. Therefore, as the trends predict many benefits for companies implementing chatbots, are we forgetting to assess whether the same systems are beneficial for its users? It would appear that we are forcing users to adopt technology which they find frustrating and useless. If the reviews find chatbot interactions as unintelligent, pointless, and not more effective than conducting a Google search or contacting a human customer service agent, what effects might this have on the user experience and the future of conversational interfaces?
This thesis project explores how we can build chatbots that offer a better user experience and are perceived as more intelligent social actors through focusing on the design of chatbot personalities. The thesis will use personality as a the stable pattern to guide the design process of a chatbot prototype, and use this personality to evaluate its effects on the user experience. This personality framework will be based on a deep understanding of users and their needs, the brand it represents, and personality theory to build an appropriate chatbot personality.

\vspace{5mm} %5mm vertical space

\section{Justification, Motivation and Benefits}
There has always been a pattern of new technologies failing to be adopted by users, because they have been released before we know how to ensure that they create more value to its users than existing solutions. Chatbots promises a lot, but fail to live up to its promises. Understanding the ways in which we can improve the user experience of chatbots can help turn this around, as we can meet user expectations and create value for users, even though there is still much to be done before chatbots truly can fulfil their potential. The one thing we can improve is how they are perceived. Understanding why chatbots are failing to be perceived as more than a computer, can have a great impact on improving the user experience. AI and NLP helps chatbots understand language, but we must take into account many other factors of human interaction beyond language understanding only, if we want to improve how they are perceived by human users. Humans have bodies, feelings, emotions, different personalities and behaviours that all influence how we communicate, behave, and interact. Chatbots needs to be able to simulate these skills for it to be perceived as more than a computer; they have to become skilled social actors. If chatbots are to become this, we must understand the social cues that make up human interaction, and the personalities that drives them. As trends show that companies are rapidly implementing chatbots as part of their services, consumers should not be forced to adopt solutions that does not improve the effectiveness, efficiency or satisfaction compared to traditional systems.

This Master's thesis aims to benefit designers of chatbots, brands that wishes to add chatbots to their services, and the users of chatbot systems. The thesis will provide this by developing, implementing, and evaluating a personality framework to help design user-centred chatbot personalities. The aim of this framework is to improve the user experience of chatbot interfaces by focusing on the personality of the chatbot, ensuring that it meets users' needs and is consistent with the brand image it represents.


\vspace{5mm} %5mm vertical space

\section{Research Questions}

The research questions and sub-questions to be addressed in the master thesis project are:

\begin{enumerate}
    \item How can we design chatbot personalities to guide the design process of a chatbot interface? 
        \subitem a) Can personality be used as a stable pattern to guide the design \subitem    process of chatbot interfaces?
        \subitem b) Which elements must be considered to inform a chatbot personality?
        \subitem c) What components needs to be in place for a chatbot personality to \subitem  meet user needs and expectations?
    \item Will chatbots with a defined personality improve the user experience of chatbot interfaces?
\end{enumerate}

\vspace{5mm} %5mm vertical space

\section{Planned Contributions}
Through this thesis the researcher hopes to contribute to the understanding of how designers can create improved chatbot interfaces, and investigate whether personality has an improved effect on the user experience. This investigation is twofold, as the first part involves developing and implementing a personality framework to build user centred chatbot personalities. This will be the first contribution that will allow other designers, developers, or researchers to use the framework and improve it. The second part of the thesis uses the modelled personality to understand whether the personality is perceived as intended, and investigate the effects personality has on the user experience of chatbots. The thesis offers insights into how thinking early about personality provides a stable pattern to the design process of chatbot interfaces. The experiment contributes to the knowledge of the effects personality has on the user experience, and also offers two methods to evaluate the personality. The first evaluation method helps determine whether users perceive the personality as intended, while the other method evaluates the personality in regards to the user experience. Both evaluation methods collects quantitative data to evaluate the chatbot personality.

\vspace{5mm} %5mm vertical space

\section{Thesis Outline}

The thesis consists of 6 chapters.\\

\textbf{Chapter 1} introduces the problem being addressed by the thesis, the justification, motivation and benefits of writing this thesis as well as the planned contributions this thesis offers to the research community and beyond.\\

\textbf{Chapter 2} provides the reader with the theory, background and existing literature regarding chatbots and conversational agents, a summary of a preliminary literature review that reviews factors that affects how humans perceive conversational agents, and personality theory to inform how to build chatbot personalities.\\

\textbf{Chapter 3} describes the methodology. This is twofold, as the first part consists of the personality framework that was used to build the chatbot prototype, while the second part includes the experiment methodology used to evaluate the modelled personality and investigate the effects the personality has on the user experience.\\

\textbf{Chapter 4} presents the results from the statistical analysis conducted to interpret the results from the evaluation of the personality and the results from the user experience evaluation.\\

\textbf{Chapter 5} includes the discussion of the implementation of the personality framework, and a discussion and interpretation of the results from the experiments assessing the personality and user experience. Possible confounders, contributions and limitations are also discussed in this chapter.\\

\textbf{Chapter 6} offers a summary and the final conclusions taken from the implementation of the framework and evaluation of the personality. The research questions, and hypotheses, will be answered in this chapter. Recommendations for future research is also offered.\\