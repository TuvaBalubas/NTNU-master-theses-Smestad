\chapter{Methodology}
\label{chap:methodology}

This project followed a user centred design methodology to structure the design process to build the chatbot prototype. The process towards building the prototype and its evaluation will be used to answer the research questions:

\begin{enumerate}
    \item How can we design chatbot personalities to guide the design process of a chatbot interface and create a better user experience? 
        \subitem a) Which elements must be considered to inform a chatbot personality?
        \subitem b)What components needs to be in place for a chatbot personality to meet user needs and expectations?
    \item How can personality be used as a design variable to allow users to perceive them as more than a computer?
        \subitem a) How can designers use personality to improve the conversational design of chatbots?
        \subitem b) How will the personality affect the user experience? %metrics consistency, differentiate from similar services, connect relationships with users, motivate - inform hypothesis
\end{enumerate}

The design process was used to investigate techniques and methods to build a user-centred chatbot persona to inform a chatbot prototype. The UCD process will be divided into the stages of:  1) inspiration 2) ideation 3) implementation (See Figure 3.0, IDEO.org, 2015). According to \cite{Gould1985} there are three key principles of UCD: an early focus on users and tasks, empirical measurement of product usage, and iterative design. The three stages will therefore all follow the key principles of UCD, where the first stage inspiration will focus on user and domain research, ideation will focus on designing the elements of the chatbot prototype, and implementation will focus on the final prototype which will be used to evaluate the design process, the personality framework, and whether personality design benefits the user experience. As a UCD approach is characterised by empirical measurement, iterative design, and focus on users, each deliverable will be empirically tested through user testing techniques in iterations to inform the final prototype.


\section{Design Process}

\vspace{5mm} %5mm vertical space

\subsection{Inspiration phase}

    \subsubsection{Brand Analysis}
    \subsubsection{Secondary Research}
    \subsubsection{User interviews}
    \subsubsection{Content Analysis}

\subsection{Ideation phase}
    \subsubsection{User persona}
    \subsection{User Scenarios}
    \subsection{Defined Requirements}

\subsection{Implementation phase}
    \subsection{Conversational Design}
    \subsection{Avatars}
    
\vspace{5mm} %5mm vertical space

\section{Personality Framework}

\vspace{5mm} %5mm vertical space

\section{Experiment Setup}

A scenario based test was constructed in order to test the personality of the chatbot and whether the personality affects the user experience. The users will be given a series of tasks testing the most important requirements of the system as revealed through the design process. The test will be used to assess whether the personality is perceived as consistent for all users by implementing the Agree! evaluation method in which compares the given characteristics and personality traits with the agreement of users. The the user experience will be measured by using the Attrakdiff questionnaire to assess the usability and design. 

\vspace{5mm} %5mm vertical space

    \subsection{Hypotheses}
 
    \begin{enumerate}
         \item \textit {H11: Chatbot A will have a positive effect on the user experience}
        \item \textit {H10: Personality has no effect on the user experience of chatbots}
        \item \textit {H21: Users will perceive chatbot A as more consistent than chatbot B}
        \item  \textit {H20: Users will not perceive chatbot A as more or less consistent than chatbot B} 
        \item \textit {H31: Users will perceive chatbot A more positively than chatbot B}
        \item \textit {H30: Users will not perceive chatbot A more or less positively than chatbot B}
    \end{enumerate}
    
    \vspace{5mm} %5mm vertical space
    
    \subsection{Data Collection}
    
    In order to collect the appropriate data from the test, the Attrakdiff questionnaire was used to assess and compare the two chatbot prototypes. The Attrakdiff assesses personal user rating of a products usability and design. It is an evaluation method that records both the perceived pragmatic quality, the hedonic quality and the attractiveness of an interactive product (cite).
    
    \begin{itemize}
        \item Pragmatic Quality: Usefulness and usability of the system
        \item Hedonic Quality: Motivation, stimulation and challenge for the user
    \end{itemize}
