\chapter{Methodology}
\label{chap:methodology}

This project followed a user centred design methodology to structure the design process to build the chatbot prototype. The process towards building the prototype and its evaluation will be used to answer the research questions:

\begin{enumerate}
    \item How can we design chatbot personalities to guide the design process of a chatbot interface and create a better user experience? 
        \subitem a) Which elements must be considered to inform a chatbot personality?
        \subitem b)What components needs to be in place for a chatbot personality to meet user needs and expectations?
    \item How can personality be used as a design variable to allow users to perceive them as more than a computer?
        \subitem a) How can designers use personality to improve the conversational design of chatbots?
        \subitem b) How will the personality affect the user experience? %metrics consistency, differentiate from similar services, connect relationships with users, motivate - inform hypothesis
\end{enumerate}

The design process was used to investigate techniques and methods to build a user-centred chatbot persona to inform a chatbot prototype. The UCD process will be divided into the stages of:  1) inspiration 2) ideation 3) implementation (See Figure 3.0, IDEO.org, 2015). According to \cite{Gould1985} there are three key principles of UCD: an early focus on users and tasks, empirical measurement of product usage, and iterative design. The three stages will therefore all follow the key principles of UCD, where the first stage inspiration will focus on user and domain research, ideation will focus on designing the elements of the chatbot prototype, and implementation will focus on the final prototype which will be used to evaluate the design process, the personality framework, and whether personality design benefits the user experience. As a UCD approach is characterised by empirical measurement, iterative design, and focus on users, each deliverable will be empirically tested through user testing techniques in iterations to inform the final prototype.


\section{Design Process}

\subsection{Inspiration phase}

    \subsubsection{Brand Analysis}
    \subsubsection{Secondary Research}
    \subsubsection{User interviews}
    \subsubsection{Content Analysis}

\subsection{Ideation phase}
    \subsubsection{User persona}
    \subsection{User Scenarios}
    \subsection{Defined Requirements}

\subsection{Implementation phase}
    \subsection{Conversational Design}
    \subsection{Avatars}

\section{Personality Framework}

\section{Experiment Setup}

    \subsection{Hypotheses}