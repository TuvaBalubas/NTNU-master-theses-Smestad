\chapter{Methodology}
\label{chap:methodology}

This project followed a user centred design methodology to structure the design process to build the chatbot prototype. The process towards building the prototype and its evaluation will be used to answer the research questions:

\begin{enumerate}
    \item Can personality be used as a stable pattern to guide the design process of chatbot interfaces?
    \item How can we design chatbot personalities to guide the design process of a chatbot interface and create a better user experience? 
        \subitem a) Which elements must be considered to inform a chatbot personality?
        \subitem b)What components needs to be in place for a chatbot personality to meet user needs and expectations?
    \item Will the personality be perceived consistently by users?
    \item How can personality be used as a design variable to allow users to perceive them as more than a computer?
        \subitem a) How can designers use personality to improve the conversational design of chatbots? %how can I operationalise conversational design and test this? - can be a simple question in the evaluation
        \subitem b) How will the personality affect the user experience? %metrics consistency, differentiate from similar services, connect relationships with users, motivate - inform hypothesis
\end{enumerate}

The design process will help answer the first research question and sub-questions. As there are no precise design methodology to build user centred chatbots, with a basis in personality, the researcher has combined techniques from user-centred design, branding, and personality theory in order to form a personality framework for chatbot interfaces. This design framework was used to build the chatbot prototype, which was tested and evaluated to answer the research questions regarding the user experience and user perceptions. The evaluation methodology will be laid out after the personality framework has been explained in the next section.

\vspace{5mm} %5mm vertical space

\subsection{Brand, Domain, and Usage}

To build the chatbot prototype, test the personality framework, and follow a user-centred design approach, the chatbot domain will be based on a real brand. There are no formal collaboration with this brand, therefore it will be anonymised in this thesis. It was necessary to use a real life example to base the prototype on in order to show how the chatbot personality represents the brand's tone of voice, mission and values. This also informed user personas, and suitable users for the chatbot prototype to model the personality on. In addition to this it also informed the role and job the chatbot should have to add value to the user and support the mission of the brand.

The chatbot prototype will be used to further the mission of the brand which is to increase the consumption of fruits and vegetables and reduce food waste. 


\subsection{User group}

The intended user group for the chatbot prototype are between the ages of 25 to 40, aiming at young parents with small or teenage children. As research have found that this age group eats less fruits and vegetables and waste the most food, the chatbot will focus on this age group. This target group usually have hectic days where healthy eating and activity can be difficult to maintain, and they are also in charge of their children’s diet and activity levels as well. As learning good habits starts when we are children, parents have a major impact in regards to teaching children the right habits. Therefore, to support the mission of the brand of increased consumption of fruit and vegetables. People in the age group 25-39 waste more food than other age groups, and in particular families with small children waste more because they have hectic schedules, and little time to plan and prepare meals. This shows that they are in need of a service which can help them cook healthy meals for themselves and their families, learn easy and economical ways to add more nutritious produce to their meals, and learn to waste less food.


\section{Design Process}

The design process was used to investigate techniques and methods to build a user-centred chatbot persona to inform a chatbot prototype. The UCD process was divided into the stages of:  1) inspiration 2) ideation 3) implementation (See Figure 3.0, IDEO.org, 2015). According to \cite{Gould1985} there are three key principles of UCD: an early focus on users and tasks, empirical measurement of product usage, and iterative design. The three stages will therefore all follow the key principles of UCD, where the first stage (inspiration) will focus on user and domain research, ideation will focus on designing the elements of the chatbot prototype, and implementation will focus on the final prototype which will be used to evaluate the design process, the personality framework, and whether personality design benefits the user experience. As a UCD approach is characterised by empirical measurement, iterative design, and focus on users, each deliverable will be empirically tested through user testing techniques in iterations to inform the final prototype.


\vspace{5mm} %5mm vertical space

\subsection{Inspiration phase}

All the stages of the design process followed a strict user-centred design methodology. The inspiration phase focused on gathering insights into the domain at hand. This included conducting a brand analysis of the available content in order to inform the mission statement, values, goals and target audiences. In addition, secondary research was gathered in order to understand trends, causes and action-plans regarding increasing healthier lifestyles and waste less food.

\vspace{2,5mm} %2,5mm vertical space

    \subsubsection{User interviews}
    
    Once secondary knowledge had been collected, a series of interviews was conducted in order to understand the experiences of the users, how do they view their intake of fruit and vegetables, what obstacles do they face, and why they think they are not eating healthy enough. Eight users were recruited, or four couples, in the ages of 29 to 36, four mothers and four fathers. All four couples had two children in kindergarten and/or early elementary school age. Six of the participants works full time, while one mothers was on maternal leave as the interviews were conducted. Two participants, one male and one female, were part-time and/or students at the time of the interviews.
    
    The interview guides prepared were semi-structured and aimed at mapping the daily routines, views, habits, pains and frustrations of the users during an average week. In particular the interviews aimed to see how parents assess their own eating and activity habits, and whether they are aware of food waste occurring and if so why. The interview guide can be found in Appendix XX.
    
    The findings from the interviews were summarised to inform a PACT analysis (appendix XX), to give an understanding of the current state of the services and digital channels of the brand in question and the user group.

\vspace{2,5mm} %2,5mm vertical space

    \subsection{Ideation phase}
    
    In the ideation phase the researcher created user personas, based on findings from the interview and the PACT analysis. To understand the requirements of the system, the user personas describes which goals the users wish to meet using the chatbot, their frustrations and pains and motivations for using the chatbot. Once the personas had been formed, findings from the inspirations phase were used to develop user scenarios --> conceptual scenarios --> concrete scenarios --> use cases. These can be found in Appendix XX.
    
    %show flowchart of how to use scenarios

    
\vspace{5mm} %5mm vertical space

\section{Personality Framework}

The design process, UCD approach, helped form the requirements for the system, define system goals, and use cases. In addition it allowed for valuable insights regarding the users, their needs and frustrations with current solutions. This was necessary to understand the role of the chatbot, and which tasks were necessary for it to do. However, while we know what it needs to do, how can we script the conversation, the different intents, in which the users will have when addressing the chatbot? 

In order to write the chatbot conversation flow, one must be able to understand how the chatbot should behave, and how this behaviour, or personality, best suits the users. To build the personality framework the researcher identified four components that the chatbot personality must be based on:

\begin{enumerate}
    \item The brand mission, goals and values
    \item A deep understanding of the users and their needs %see whether the inspiration phase should be located here
    \item The role/job of the chatbot
    \item An appropriate personality model
\end{enumerate}

These four conditions describes four questions that designers must have the answer to in order to have the necessary components to build the chatbot personality.

\vspace{5mm} %5mm vertical space

    \subsection{The Brand Mission, Goals, and Values}
    To design a chatbot that conforms to the brand it represents, the brand image was analysed and defined. The mission statement was identified, the company and core values was defined, and a tone-of-voice analysis was conducted.
    
\vspace{2,5mm}
    
        \subsubsection{Mission statement and core values}
 
        The brand describes itself as having a broad social responsibility. They supply fruits and vegetables, with suppliers on all continents providing products from all over the world. Freshness is the top priority and high quality. In addition to providing fresh produce, the brand recognises having a fair and sustainable business model, which extends to proper nutrition, increased activity, and societal and environmental responsibilities. Their goal is to increase the consumption of fruit and vegetables, therefore in addition to supplying fresh produce, they take an active role in increasing knowledge and engagement regarding healthy lifestyles and positive societal development.
 
\vspace{2,5mm}
  
        \subsubsection{Mission:}
        Fresher and healthier 
        Taken a society position in the  market that contributes to increased focus on healthy diets and physical activity
        A driver and pioneer for sustainable development regarding the environment and societal responsibility within the company’s product area
        
\vspace{2,5mm}

        \subsubsection{Values and internal conduct:}
        
        To create value for our consumers/customers is our driving force throughout the business. The conduct of employees are characterised by the values:
        
        \begin{enumerate}
            \item Team spirit
            \item Courage to do the right thing: openness, honesty, probity
            \item Compassion: mutual respect, care, and tolerance
        \end{enumerate}

\vspace{2,5mm}

        \subsubsection{Core Values:}
        \begin{enumerate}
            \item Vital
            \item Current
            \item Fresh
            \item Daring
        \end{enumerate}
    
\vspace{5mm} %5mm vertical space

        \subsubsection{Tone of Voice analysis}
    
        If the chatbot are to be perceived as an extension or continuation of the brand, it must adhere to the brands tone-of-voice. Tone-of-voice is defined as a brands personality, communicated through its written communication as well as the visual communication. Norman \& Nielsen defined a framework to determine tone-of-voice based on 4 dimension: funny vs. serious, formal vs. causal, respectful vs. irreverent, enthusiastic vs. matter-of-fact.
    
        \begin{figure}
            \centering
            \includegraphics[width=\textwidth]{figures/Tone-of-voice.png}
            \caption{This figure shows how the tone-of-voice was defined by understanding the mission, the values and conducting a tone analysis of the written copy}
            \label{fig:tov}
        \end{figure}
    
        Figure \ref{fig:tov} shows how the tone-of-voice was determined by assessing the mission statement, the company's values inwards and outwards, and through a tone analysis. The tone analysis was conducted using the IBM tone analyser tool, the analyser assessed the written copy found on the brands website. Their tone-of-voice as reflected in their copy is more serious than funny, but not too serious as the tone is also a nice in-between of formal and casual. Their tone is very respectful, which follows their core values of tolerance and respectfulness. Lastly they are leaning towards more enthusiastic than matter-of-fact as they carry a very positive and joyful view of the future and their path to fulfil their goals.

\vspace{5mm}

    \subsection{The Chatbot Role}
    
    The user-centred approach identified the two most important jobs the chatbot could do that both furthers the goals of the brand and add value to the users: increase the consumption of fruits and vegetables, and reduce food waste. Therefore, the job of the chatbot agent is to \textit{assist} users to help them take healthier choices for their family and \textit{motivate} them to successfully implement these changes. The chatbot is there to assist and guide the users, with the long term goal of successfully implementing a fresher and healthier lifestyle and reduce food waste. 
    
    To achieve this the chatbot will help parents plan their dinners for the whole week, assist with the shopping and make recommendations based on ingredients they have in house and leftovers. In addition to this the chatbot encourages parents to inform the chatbot about what they have eaten so far, what was a success and what wasn’t so that the chatbot can learn which items are not eaten, are wasted and instead offer healthy alternatives that will be eaten rather than wasted. 
    
    Assistants and motivators have specific traits in which they need to have in order to be successful in their role:
    
\vspace{2,5mm}
    
    \textbf{Motivator}
    \begin{enumerate}
        \item Give praise and encouragement
        \item Treat clients as equals
        \item Show trust
        \item Communicate and set goals
        \item Be attentive
        \item Allow mistakes
        \item Be pleasant 
        \item Ask for feedback
        \item Keep others informed
        \item Don’t micromanage
    \end{enumerate}
    
    \textbf{Assistants}
    \begin{enumerate}
        \item Professionalism
        \item Collaborators
        \item Outstanding organisational skills
        \item Excellent communication skills
        \item Willingness to go the extra miles
        \item Problem-solver
        \item Proactive
        \item Respectful
    \end{enumerate}
    
 
    The traits of the assistant will be reflected in the way the chatbot handles its job. Through AI and NLP capabilities, as well as connection to necessary APIs, will the chatbot be able to complete necessary tasks and incorporate “hidden/invisible” features that will help users achieve their goals. As for the motivator traits, this will be reflected in how it encourages/motivates users, the language its using, and its behaviour (prompts, affirmations, tips). Therefore the chatbot role will be reflected by its external traits (motivator) and its internal traits (assistant).
    
\vspace{5mm}

    \subsection{Personality Trait Model}
    Once the brand mission, core values, understanding of user needs, and the chatbot role have been identified, designers should find an appropriate personality trait model to help put together a dynamic personality for the chatbot. It is important that this model will be used to place the desired personality within a framework, this is to benefit the designer when designing a consistent personality. 
    
    It is not the goal of this thesis to assess whether the designed personality is consistent with the chosen trait model, or to assess the personality of participants. The trait model is only used to guide the design of the chatbot personality, and to help in evaluating desirable traits for the specific chatbot agent. 
    
    The chosen personality model to model this chatbot's personality was the five factor model. As explained in section 2.3.2, this personality trait model is widely know, and based on lexical data used by humans to describe personalities. These traits have been mapped out and organised into the five factors, and as the traits are identified by the words we use to describe them, it will be an appropriate model for a linguistic interface such as a chatbot.
    
\vspace{5mm}

        \subsection{Prototype \& Conversational Design}
        The chatbot prototype was built using the Chatfuel bot builder platform. The conversation design was written by mixing a user scenario technique with mind-mapping tools, such as Xmind, to map out the different chatbot skills, user intentions, and conversation flows.
        
\vspace{5mm}

        \subsection{Avatars}
        The literature review investigated how humans perceive the various types of conversational agents, and found that anthropomorphism can benefit how a personality is perceived, and that a consistent personality dictates which characteristics humans assign to a chatbot in which they anthropomorphise. The literature review also discussed humanness as it related to human capabilities and physical characteristics. The review found that chatbots with a high levels of humanness increases trust and familiarity, while also providing a more natural interaction with humans. The latter is explained by that humans know how to interact with other humans, therefore will assume that a chatbot will interact like a human if looking human. 
        
        Because of this, the chatbot avatar will have a human appearance. This chatbot is taking on human tasks and is to be perceived as a human. However, research have found that too human might have a negative effect on human perception which is why the chatbot will be portrayed as an illustrated character rather than a realistic human. This will make clear the distinction that the chatbot is not a human, while incorporating high levels of humanness simultaneously. 
    
\vspace{5mm} %5mm vertical space
    
\section{Experiment Setup}
    
Participants will evaluate the two chatbots by conducting a scenario and task based test, in order to test the personality of the chatbot and whether the personality affects the user experience. The users will be given a series of tasks testing the most important requirements of the system as revealed through the design process. The test will be used to assess whether the personality is perceived as consistent for all users by implementing the Agree! evaluation method in which compares the given characteristics and personality traits with the agreement of users. The the user experience will be measured by using the AttrakDiff questionnaire to assess the hedonic vs. pragmatic quality of the interface. To truly assess whether the personality have an effect on user's perception and user experience, participants will be interacting with two chatbots. The two chatbots are exactly the same in all parts excepts their personalities. While one of the chatbots will display the identified personality built through the personality framework, the other will not display a human-like personality but rather act like a machine. %show examples of how you have defined "not having a personality"
    
\vspace{5mm} %5mm vertical space

    \subsection{Hypotheses}
 
    \begin{itemize}
         \item \textit {H11: Chatbot A will have a positive effect on the user experience}
        \item \textit {H10: Personality has no effect on the user experience of chatbots}
            \vspace{5mm} %5mm vertical space

        \item \textit {H21: Users will perceive chatbot A as more consistent than chatbot B}
        \item  \textit {H20: Users will not perceive chatbot A as more or less consistent than chatbot B} 
            \vspace{5mm} %5mm vertical space

        \item \textit {H31: Users will perceive chatbot A more positively than chatbot B}
        \item \textit {H30: Users will not perceive chatbot A more or less positively than chatbot B}
    \end{itemize}
    
    \vspace{5mm} %5mm vertical space
    
    \subsection{Data Collection}
    
    \vspace{5mm} %5mm vertical space

     \subsubsection{Agree!}
     
     In order to evaluate whether the personality is perceived consistently, the participants will be asked to describe the personality in relation to predefined characteristics that are compatible with the chosen personality. The participants will be given a set of characteristics and rate them on a dimension to determine whether they perceived that characteristic more or less when interacting with the chatbot. The evaluation will be conducted using the Agree! Tool. This is a software developed by Callejas (2014) that implements the framework for the assessment of synthetic personalities. %site website. 
     
     According to Callejas (2014) Agree! will help evaluate personality in three main dimensions:
    
        \begin{itemize}
            \item Whether the rendered personality is perceived by the users as the designers intended. For example, if the designers plan an extrovert personality, whether users perceive it as extrovert or as something else.
            \item Whether the personality is recognizable, that is, if users perceive it consistently (i.e. if users agree in their perceptions, or different users perceive very different personalities).
            \item Whether the agent's personality matches the users' personality, and how the previous dimensions are affected by the personality of users.
        \end{itemize}
            
    \vspace{5mm} %5mm vertical space
   
     \subsubsection{AttrakDiff}
    
    In order to collect the appropriate data from the test, the AttrakDiff questionnaire was used to assess and compare the two chatbot prototypes. The AttrakDiff form assesses personal user rating of a products usability and design. It is an evaluation method that records both the perceived pragmatic quality, the hedonic quality and the attractiveness of an interactive product (cite).
    
        \begin{itemize}
            \item Pragmatic Quality: Usefulness and usability of the system
            \item Hedonic Quality: Motivation, stimulation and challenge for the user
        \end{itemize}
        
    The data collection forms will be presented to participants after they have interacted with each chatbot. The participants will be presented with the same form for each bot, and the forms will be accessed locally through the researchers private computer in order to ensure participants anonymity. 
    
   \subsubsection{Task Based Experiment}
   
   As stated earlier in this section the participants will be taken through different scenarios with specified tasks in order to interact with both chatbot systems. Half of the group will test chatbot A first, while the second half will test chatbot B first. This is to avoid reactivity where users will become familiar with the interface, and therefore feel more at ease when interacting with the second chatbot. 
    

   