\chapter*{abstract}

In 2017 we see a rise in chatbot technology being made available on the web and on our mobile devices, and recent reports states that 56 \% of companies have implemented or are planning to implement a chatbot in the near future. Chatbots are therefore a big part of an AI powered future, however recent reviews find chatbots to be perceived as unintelligent and non-conversational. Such findings have not slowed down the rapid implementation of chatbots online, and the same mistakes seems to be repeated over and over again. Chatbot services have been found to save companies and estimate of \textdollar 8 billion by 2022, and extends to customer service tasks, product purchasing, shopping assistants, recommender systems, service or product support. This explains why so many are eager to implement their own chatbots, but the reviews make one wonder whether we now are forcing users to adopt technology which they find frustrating and useless. Chatbots are becoming an extension of the services companies provide, therefore ensuring a great user experience is important not only for a company's brand image, but also for the users of their services. Existing literature regarding how humans perceive conversational agents have found that personality can offer a stable pattern to how the chatbot is perceived, and add consistency to the user experience. Chatbots that include a well-defined personality are more likely to be perceived as having a mind of its own. This project will investigate how we can improve the user experience of chatbots through a user-centred approach to design chatbot personalities. The researcher plans a project in which she will design a chatbot personality and use this personality to guide the design process of a chatbot prototype. The goal of developing this prototype is to investigate how this approach will benefit both the design process and the user experience, and to create a UX framework which will guide the design process towards user-centred chatbot interfaces.

\hypersetup{pageanchor=false}