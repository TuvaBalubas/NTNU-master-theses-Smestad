\chapter{Discussion}
\label{chap:discussion}

Discussing the design process: what worked? what didn't - how could it improve? Agree framework e.g.other challenges in the design process? CHatfuel as a tool, how did it work where was it limited - answer research question 1 and sub questions - did it help?

\section{Discussion: The implementation of the personality framework}

The first research question asks \textit{How can we design chatbot personalities to guide design process of a chatbot interface?}, and the sub-questions concerns whether personality can be used as a stable pattern to guide the design process, which elements must be considered to inform the personality, and how can we ensure that the personality meets user needs and expectations. Through following the personality framework, the researcher found that the elements that must be considered are as follows:

\begin{enumerate}
    \item A deep understanding of the users and their needs
    \item The brand mission, goals, and values
    \item The role/job of the chatbot
    \item An appropriate personality model
\end{enumerate}

These four elements were all used to form the personality of the chatbot, and helped narrow down the most important tasks the chatbot should perform, as well as understanding the way in which these tasks should be performed. As laid out in the methodology section, chatbots that are extensions of services provided by brands, must be consistent to the brand it represents. By having a deep understanding of the brand guidelines; tone of voice, values, mission and goals, will limit and set precise guidelines for the way in which the chatbot should behave. In addition to how it should conduct itself, it will also have to be built to understand how it should handle the different users regarding their emotions, needs, expectations, and contexts. This specific understanding not only adds to its behaviour, but also allows designers to plan for which characteristics should be present at what time. If the users are frustrated or in a hurry, the chatbot should not behave in a way that adds to their frustrations or become more time consuming. Understanding these helps write user scenarios, and will add to the knowledge of the chatbot, as different user scenarios needs different use cases. Therefore, designers have already predicted and planned for use cases that the chatbot might have to handle. Personality plays into this in regards to how it is designed to respond or act in these scenarios. The chatbot built for this thesis were built on the knowledge that its users will not respond well to a lecturing tone, the "I-know-best", and therefore took a much different approach to providing help and tips. The last point here also consists of the role of the chatbot, the job its performing again limits the way in which it should behave. This chatbot was defined not only as an assistant, but a motivator as well. It was there to support, guide, and help, not to lecture, challenge, and tell - which might be a suitable approach for a different context and a different target audience. This shows how important it is to not choose a personality before we know 1) who the users are 2) which values it represents 3) which job it is performing! This should form the basis of the personality, and not come after-the-fact. Then you need to find an appropriate personality model. 

This project chose to use the five-factor model, to summarise why this model and not e.g. MBTI, HEXACO or Jung, its because this model has been found to be suitable for multiple countries and cultures, and provides a descriptive taxonomy, in which organises natural language and the scientific traits into a singe framework. It is therefore an organised and easy framework to use when wanting to build a personality. The model is created to assess persons personality types, but can very well be reversed to build a suitable personality; as done in this thesis project. Having the list of traits, organised under the five factors, was helpful to use throughout the scripting of the chatbot. It was a helpful tool when writing each use case, and plan for any conversation flow. It was also very helpful when building the personality for Chatbot B.

Why not give it a flaw? Why did it not display any of the neurotic traits?

The results you have collected and the process you when through to develop the project have been presented earlier.  This Chapter is used to talk about your interpretations of results or the process.  It might be a discussion of the language you used.  A tool that you started to use but then stopped using for some reason.  It could give insight into the evolution of your process.

There was only one word-pair in which Chatbot B scored higher than Chatbot A \textit{unprofessional-professional}, where the mean score for Chatbot A was 5,875 and Chatbot B mean score was 6,25. This shows that a machine like and technical Chatbot B were perceived as conducting itself to a higher professional standard than Chatbot A, ever so slightly.

\section{Discussion: The experiment}

\subsection{Regarding Facebook and the Cambridge Analytica scandal}

Because of the Facebook and Cambridge Analytica scandal during the spring of 2018; Facebook stopped reviewing any third party applications running through their platform. This resulted in the Chatbot B prototype not being able to be published through the messenger platform in time for the experiments. Chatbot A was able to be published through an existing application already reviewed by Facebook before the scandal, however the name of the chatbot was unable to be changed in time of the experiment.

\subsection{Possible confounders}
Because Facebook stopped reviewing new third party applications during the experiment, Chatbot B did not display a profile picture, but instead the chatfuel logo, and the name was displayed as "Chatfuel-test your bot", as this had to be tested through the unpublished test service provided by Chatfuel, rather than being run through its own page and account. Chatbot A however was runned through its own page and account, but the name of the chatbot belonged to a previous version of a chatbot displaying the wrong name. The absence of an image and name for Chatbot B could have impacted how users perceived this chatbot. Some unconsciously referred to this chatbot as a "he" rather than "she" for Chatbot A - even though all participants were instructed before beginning the tests that they are about to test two versions of the chatbot "Bella". The cards each participants used during the test had Bella's face on them in order to remind them of who they are talking to. 

Another possible confounder was one that the researcher failed to recognise before during the testing. This was the recipes suggested by the chatbot. A few of the test participants told the researcher after the experiment had finished, that they rated e.g. the word pair \textit{inventive-conventional} higher because the recipes in themselves were inventive - or they rated the characteristic helpful lower because they did not like the recipes suggested to them. The participants were not instructed to assess the personality of the chatbot, but the chatbot as a product in its entirety, therefore a few of the word-pairs assessed in the AttrakDiff measurement tool might be seen as negative for a product, but positive for a personality e.g. \textit{undemanding-challenging}, were \textit{challenging} was seen as a negative trait to describe the agent by some while for the assessment of a product it is seen as a positive trait. Another word-pair that confused participants were \textit{separates me from people - brings me closer to people}, as a lot of participants noted that talking to a robot would in the long run separate them from people rather than bringing them closer.


Building a personality framework would be useless, if personality did not have an improved effect on the user experience. This is why this project is twofold 1) building, implementing, and testing a personality framework for chatbot interfaces, 2) testing whether personality improves the user experience of chatbot interfaces. The second would be difficult to test without a personality framework, and the framework would be useless if the result of implementing it did not in any way affect the user experience.

\section{Contribution}

\section{Limitations}

