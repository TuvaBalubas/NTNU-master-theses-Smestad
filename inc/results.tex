\chapter{Results}
\label{chap:results}

This section will provide the results from the data analysis of both the test of the agreement regarding the personality traits and characteristics, and the AttrakDiff evaluation to assess the user experience. The data was analysed using SPSS.

\section{Results from personality characteristics data analysis}

The characteristics were rated using a scale from 1-5 in which 5 meant a high presence of the characteristic in the interaction with bot, and 1 meant no presence of the characteristics. The data-set was analysed through a two-way repeated measures ANOVA to investigate whether an interaction effect occurred within subjects, as all participants were subjected to all conditions.

The descriptive statistics of the characteristics data analysis can be found in \todo{write in statistics descriptive}. 

\todo{add the descriptive statistics table}

A two-way repeated measures ANOVA was conducted that examined how the participants perceived the appointed personality traits to each chatbot version. The independent variable personality has two levels, and the dependent variable characteristics has four levels where each level describes a factor from the five factor model (extroversion, agreeable, conscientiousness, openness). The two-way repeated ANOVA included a between subject factor (startbot) to investigate whether the order of chatbots participants were subjected to had any effect on the results.

There was a significant main effect of the two levels of personality with respect to the characteristics F(1,14)=73,181, p<,001 and there was an INsignificant interaction effect between personality and the start bot p=,926. There was also a significant main effect between the characteristics F(3,42)=12,960, p<,001, and an INsignificant effect between characteristics and start bot p=,869. There was a significant interaction effect between personality and characteristics p<,001. There was an INsignificant interaction effect between the personality and characteristic and the starting order (personality*characteristics*startbot) p=,380.

\section{Results from AttrakDiff data analysis }

The data-set was analysed through a two-way ANOVA to understand whether an interaction effect occurred that could have an impact on how users rated either version. The analysis investigated the effect the independent variable personality had on the dependent variable user experience. Personality includes two levels, and user experience is made up of four factors (pragmatic, hedonic identity, hedonic simulation, attractiveness).

The descriptive statistics of the AttrakDiff data analysis can be found in \todo{write in statistics descriptive}. 

\todo{add the descriptive statistics table}

A two-way repeated measures ANOVA was conducted to investigate the effect of the starting condition (startbot) had on the independent variable personality and the dependent variable user experience. Personality has two levels (personality, no personality) and user experience has four factors (pragmatic, hedonic stimulation, hedonic identity, attractiveness). The between-subjects factor (startbot), tested for any interaction effect dependent on which chatbot participants started with (chatbot A or Chatbot B).  

There was a significant main effect of the two levels of personality with respect to the user experience F(1,14)=15,300, p=,002), and the between subject factor (startbot) had no significant interaction effect on personality p=,847. There was a significant main effect of the user experience F(3,42)=12,264, p<,001, and the start condition (startbot) had no significant interaction effect on user experience p=,865. There was no significant interaction effect between personality, user experience and start condition (personality*UXscore*startbot) p=,909.

The paired samples t-test found that there is a significant difference in the scores between Chatbot B and Chatbot A, where all four factors of the user experience showed a significant improved effect between Chatbot B and A. Pragmatic Quality Chatbot B (M=5,4732, SD=,59649) and Chatbot A (M=5,9286, SD=56424); t(15)=-2,152, p = ,048. Hedonic-I Quality Chatbot B (M=4,7679, SD=,56874) and Chatbot A (M=5,4821, SD=,53165) t(15)=-3,239, p = ,006. Hedonic-S Quality Chatbot B (M=4,7768, SD=1,19405) and Chatbot A (M=5,6250 SD=,29909) t(15)=-2,934, p = ,010. Attractiveness Chatbot B (M=5,339, SD=,7805) and Chatbot A (M=6,3482, SD=,44864) t(15)=-4,069, p = ,001. These results suggests that personality has an improved effect on the user experience of chatbots, as all four factors of user experience was scored higher for chatbot A than chatbot B.

The pairwise comparisons table showed that there was a significant difference between the two levels of personality. With a mean difference ,757 between chatbot A and Chatbot B (p=,001).

There is often the need to layout tables that are long.  We suggest using the sidewaystable mode for this.  You will need to manage the total width of the table to make it fit, but you can see how to do that in the example in Table~\ref{tbl:CBM-MC}.

Writing your own \LaTeX{} tables is very slow, and error prone. There are many converstion tools to help create tables.  The one we used for the sideways table, Table~\ref{tbl:CBM-MC} is LatexKit\footnote{Google plugin for LatexKit \url{https://chrome.google.com/webstore/detail/latexkit/piadpbgaacpbaicjilhfebbfgofomiic}}

% you can also set widths using a new command 
%\newcommand{\Colwidth}{0.08\textwidth}
%\begin{tabular}{l|p{\Colwidth}|p{\Colwidth}|p{\Colwidth}|p{\Colwidth}|p{\Colwidth}|p{\Colwidth}|p{\Colwidth}|p{\Colwidth}|}

\begin{sidewaystable}
\small{
\begin{tabular}{l|cccccccccccccccccc}
Candidate & 1 & 2 & 3 & 4 & 5 & 6 & 7 & 8 & 9 & 10 & 11 & 12 & 13 & 14 & 15 & Num & CBM & Diff. \\
\hline
10001 & 0 & -0.5 & 1 & -0.5 & 1 & -0.5 & -0.5 & 2 & 2 & -2 & 2 & 1.5 & -0.5 & 2 & 2 & 8 & 9 & 0.5 \\
10002 & 0 & 2 & 0 & 0 & 0 & 2 & 2 & 2 & 1 & 1 & 1 & 2 & 0 & 2 & 2 & 10 & 17 & 4.5 \\
10003 & 0 & 2 & -0.5 & 0 & 0 & 2 & 1.5 & 1.5 & 2 & -0.5 & 2 & 1 & 0 & 1.5 & 1.5 & 9 & 14 & 3.5 \\
10004 & 2 & -0.5 & 1 & 0 & 0 & 0 & 2 & 2 & 1 & 0 & 0 & 1 & -0.5 & 1 & -0.5 & 7 & 8.5 & 1.5 \\
10005 & 1 & 0 & 0 & 0 & 0 & 1 & 1 & 0 & 1 & 1 & 1 & 1 & 0 & 1 & 1 & 9 & 9 & -1.5 \\
10006 & 2 & -2 & 0 & -0.5 & 1.5 & 1 & 0 & 2 & 2 & 0 & 2 & 2 & -0.5 & 2 & 2 & 9 & 13.5 & 3 \\
10007 & 1 & -2 & 0 & 0 & 1 & 0 & 1.5 & 2 & 2 & -2 & 0 & 1.5 & -2 & 2 & 1 & 8 & 6 & -2.5 \\
10008 & 2 & 2 & -2 & 0 & 2 & 0 & 2 & 2 & 1.5 & 1.5 & 2 & 1.5 & -0.5 & 2 & 2 & 11 & 18 & 3.5 \\
10009 & 2 & -0.5 & 1 & 0 & -0.5 & 2 & -0.5 & 2 & 2 & 0 & 2 & 2 & -0.5 & -0.5 & 2 & 8 & 12.5 & 4 \\
10010 & 2 & -2 & 2 & 0 & -2 & 2 & 2 & 2 & 1.5 & 1.5 & 1.5 & 2 & 0 & 2 & 2 & 11 & 16.5 & 2 \\
10011 & 2 & 2 & 0 & 1 & 2 & 2 & 2 & -0.5 & 2 & 0 & 2 & 2 & -0.5 & 2 & 2 & 11 & 20 & 5.5 \\
10012 & 2 & 1 & 0 & 2 & -2 & 2 & 2 & 2 & 2 & 0 & 1 & 2 & 0 & 2 & 2 & 11 & 18 & 3.5 \\
10013 & 2 & 2 & 1.5 & 0 & 0 & 2 & 2 & 2 & 1 & 0 & 1 & 2 & -0.5 & 2 & 2 & 11 & 19 & 4.5 \\
10014 & 2 & 2 & 1 & -0.5 & -2 & -2 & 1.5 & 2 & 2 & 1.5 & 2 & 2 & 2 & -0.5 & 2 & 11 & 15 & 0.5 \\
10015 & 1.5 & -2 & 1.5 & 1.5 & 2 & 0 & 2 & 2 & 1.5 & 1 & 2 & -0.5 & 1 & 0 & 1.5 & 11 & 15 & 0.5 \\
10016 & 1 & 2 & 0 & 0 & 1.5 & 0 & 0 & 2 & 1 & 0 & 1 & 1.5 & -0.5 & 1 & 2 & 9 & 12.5 & 2 \\
10018 & 2 & 2 & 1 & 0 & -2 & 0 & -2 & 1.5 & 2 & 1 & 0 & 1.5 & 0 & 1 & 2 & 9 & 10 & -0.5 \\
10019 & 2 & 2 & -2 & 0 & 1.5 & 2 & -0.5 & -0.5 & 1.5 & 1 & 2 & 2 & 0 & 1 & 2 & 10 & 14 & 1.5 \\
10020 & 2 & 2 & 0 & 0 & 2 & 2 & 2 & 2 & -0.5 & 1 & 2 & 2 & 2 & 2 & 2 & 12 & 22.5 & 4.5 \\
\end{tabular}
}
\caption[Confidence Based Marking]{Multichoice marking example data with 15 questions, Number correct (Num), the Confidence Based Marking (CBM) grade,  and the differentiation of knowledge quality (Diff)}
\label{tbl:CBM-MC}

\end{sidewaystable}

%\begin{sidewaystable}
%\centering
%  \csvautobooktabular{figures/largeTable.csv}
%\caption[Autogenerated on Sideways page]{using CSV tables to autogenerate the table in a sideways table. This is not an appropriate use of a sideways table as it is small}
%\label{tbl:dataSetsideways}
%\end{sidewaystable}
