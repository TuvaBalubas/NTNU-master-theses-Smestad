\chapter{Conclusion}
\label{chap:conclusion}

This thesis aimed to investigate how we can design user centred chatbot personalities and whether personality matters for the user experience of chatbots. The latter has been a long standing assumption, and based on findings related to other interfaces than chatbots or other conversational interfaces. While findings regarding traditional web and app user interfaces have found that personality is an important factor for how users perceive the interface, no similar study has yet investigated whether the same is true for chatbots. If personality does not matter, then we do not need to spend time, money and effort to create personalities for the chatbots, and can spend more time on developing a seamless experience with great and effective task handling. However, the findings presented in this thesis suggests that a human personality not only improves the user experience compared to chatbots without a human personality, it also impacts a users perception overall. Even though Chatbot A and Chatbot B provided the same service, just as effectively and efficiently, Chatbot A was still rated higher in the pragmatic quality than Chatbot B. This shows that an overall great user experience also improves how users perceive other qualities of the chatbot that are not in reality better than other versions.

The findings from the statistical analysis of the personality characteristics evaluation found a significant difference between the four factors (Extroversion, Agreeableness, Conscientiousness, Openness) for Chatbot A and Chatbot B. There was no significant interaction effect between the starting condition and gender. The results found that Chatbot A was rated higher on all factors than Chatbot B. We can therefore keep our research hypothesis $H_1 1$ that users will perceive the personality of Chatbot A as different to the personality of Chatbot B. As for $H_1 2$ and $H_1 3$ in that users will perceive both Chatbot A as intended and Chatbot B as intended, we can say to some extent that they were perceived as intended. However the researcher expected a lower score for Chatbot B in the two factors: Openness and Agreeableness. The extroversion factor was perceived as intended with Chatbot A receiving a very high mean score of 4,3125 and Chatbot B a very low mean score of 2,2708. The conscientious factor was assumed to be rated more or less equal for both chatbots, and the mean scores differed by 0,375, confirming to some extent with the researcher's assumptions. We can conclude that Chatbot A was perceived as intended with high scores on all factors, while Chatbot B were perceived as intended in the Extroversion and Conscientiousness factors, and not for the Agreeable and Openness factors as the intention was for users to perceive them to a much lower degree. 

The findings from the statistical analysis of the results of the AttrakDiff data found there to be a significant difference between the pragmatic quality, hedonic quality and attractiveness of Chatbot A and Chatbot B. The two-way repeated measures ANOVA found a significant main effect between the two personalities with respect to the user experience. The results of the paired-samples t-test found that there was a significant improved effect on the user experience of Chatbot A compared to Chatbot B. Chatbot A was rated to a higher degree in all factors, pragmatic quality, hedonic quality and attractiveness, over Chatbot B. There was no significant interaction effect found between the starting condition with respect to the personality or user experience score, nor was there found a significant interaction effect of gender. Therefore, based on this analysis, we can keep our research hypothesis $H_1 4$ that personality affects the user experience of chatbots and $H_1 5$ that Chatbot A will have an improved effect over Chatbot B. We must however remember that the confidence intervals showed that some of the scores showed a large variance in the upper and lower bounds, indicating that the significance found in the paired samples t-test should be tested further with a larger sample size to be read as more reliable.

\section{Future Work}
\label{sec:future}
\section{Future research}

It is recommended to use the Agree! framework \citep{Callejas2014} more actively when testing the personality. The framework is built on participants interacting with a chatbot in a series of interactions where there is a change in attitude (e.g. unfriendly, neutral, friendly). Instead of testing the change in attitude in this project, participants will instead be asked to rate on a Likert scale to what extent they perceived the specific traits.

For future research it would be recommended to test the effects of the personality over time.
