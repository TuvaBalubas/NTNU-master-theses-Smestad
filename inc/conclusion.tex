\chapter{Conclusion}
\label{chap:conclusion}

This is where you provide an overview of the thesis now that it is finished.  What are the critical things that can be learnt from the thesis for the reader.

The findings from the statistical analysis of the personality characteristics evaluation found a significant difference between the four factors (extroversion, Agreeableness, Conscientiousness, Openness) for Chatbot A and Chatbot B. There was no significant interaction effect between the starting condition and gender. The results found that Chatbot A was rated higher on all factors than Chatbot B. We can therefore keep our research hypothesis $H_1 1$ that users will perceive the personality of Chatbot A as different to the personality of Chatbot B. As for $H_1 2$ and $H_1 3$ in that users will perceive both Chatbot A as intended and Chatbot B as intended, we can say to some extent that they were perceived as intended. However the researcher expected a lower score for Chatbot B in the two factors: Openness and Agreeableness. The extroversion factor was perceived as intended with Chatbot A receiving a very high mean score of 4,3125 and Chatbot B a very low mean score of 2,2708. The conscientious factor was assumed to be rated more or less equal for both chatbots, and the mean scores differed by 0,375, confirming to some extent with the researcher's assumptions. We can conclude that Chatbot A was perceived as intended with high scores on all factors, while Chatbot B were perceived as intended in the Extroversion and Conscientiousness factors, and not for the Agreeable and Openness factors as the intention was for users to perceive them to a much lower degree. 

The findings from the statistical analysis of the results of the AttrakDiff data found there to be a significant difference between the pragmatic quality, hedonic quality and attractiveness of Chatbot A and Chatbot B. The results also indicated that Chatbot A was rated to a higher degree in all factors than Chatbot B. There was found not significant interaction effect between the starting condition or gender. Therefore, based on this analysis, we can keep our research hypothesis $H_1 4$ that personality affects the user experience of chatbots and $H_1 5$ that Chatbot A will have an improved effect over Chatbot B. We must however remember that the confidence intervals showed that some of the scores showed a large variance in the upper and lower bounds, indicating that the significance found in the paired samples t-test should be tested further with a larger sample size to be read as more reliable.

\section{Future Work}
\label{sec:future}
\section{Future research}

It is recommended to use the Agree! framework \citep{Callejas2014} more actively when testing the personality. The framework is built on participants interacting with a chatbot in a series of interactions where there is a change in attitude (e.g. unfriendly, neutral, friendly). Instead of testing the change in attitude in this project, participants will instead be asked to rate on a Likert scale to what extent they perceived the specific traits.
