%\thesistitlepage % make the ordinary titlepage
\hypersetup{pageanchor=false}
%\include{summary}

\chapter*{Preface}
This master thesis is the final part of my Master in Interaction Design degree at the department of Design at the Norwegian University of Science and Technology (NTNU). The project planning and preliminary studies/literature review was conducted during the autumn of 2017. The work presented in this thesis was conducted and written during the spring of 2018 and the workload corresponds to 30 ECTS.

Throughout my work as an interaction designer I was challenged with building my first chatbot and was surprised to find that much work is still to be done regarding the user experience of chatbots. Most of the frameworks acts more like "do's" and "dont's" rather than based on research. And a lot of these guidelines are based on assumptions, or small case studies that do not necessarily extends to the larger application of conversational agents. Not wanting chatbots to be abandoned early by users because the implementation of these systems are far behind the technological development, I set out with this thesis to try and prove at least one assumption: does personality improve the user experience of chatbots?

I aim with this master thesis to add to the field of human-computer and human-robot interaction by providing evidence to explain how personality impacts the user experience of machines that can converse. In addition to this I'm also providing other designers and developers, interested in conversational interfaces, with a framework to build synthetic personalities for chatbot agents.

This thesis is for anyone who wishes to design their own chatbot with a basis in personality, and those who like me are eager to determine how we can improve the user experience of conversational interfaces.

\vspace{5mm}

%\begin{center}
%\thesiscampus, 
\thesisdate \\[1pc]
\\[1pc]
%\thesisauthor
%\end{center}